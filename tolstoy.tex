\documentclass{article}
\usepackage{graphicx}

\begin{document}

\title{War and Peace}
\author{Leo Tolstoy}

\maketitle

\begin{abstract}
The abstract text goes here.
\end{abstract}

After receiving her visitors, the countess was so tired that she gave
orders to admit no more, but the porter was told to be sure to invite to
dinner all who came “to congratulate.” The countess wished to have
a tête-à-tête talk wih the friend of her childhood, Princess Anna
Mikháylovna, whom she had not seen properly since she returned from
Petersburg. Anna Mikháylovna, with her tear-worn but pleasant face,
drew her chair nearer to that of the countess.

“With you I will be quite frank,” said Anna Mikháylovna. “There
are not many left of us old friends! That’s why I so value your
friendship.”

Anna Mikháylovna looked at Véra and paused. The countess pressed her
friend’s hand.

“Véra,” she said to her eldest daughter who was evidently not a
favorite, “how is it you haves so little tact? Don’t you see you are
not wanted here? Go to the other girls, or...”

The handsome Véra smiled contemptuously but did not seem at all hurt.

“If you had told me sooner, Mamma, I would have gone,” she replied
as she rose to go to her own room.

But as she passed the sitting room she noticed two couples sitting,
one pair at each window. She stopped and smiled scornfully. Sónya was
sitting close to Nicholas who was copying out some verses for her, the
first he had ever written. Borís and Natásha were at the other window
and ceased talking when Véra entered. Sónya and Natásha looked at
Véra with guilty, happy faces.

It was pleasant and touching to see these little girls in love; but
apparently the sight of them roused no pleasant feeling in Véra.

“How often have I asked you not to take my things?” she said. “You
have a room of your own,” and she took the inkstand from Nicholas.

“In a minute, in a minute,” he said, dipping his pen.

“You always manage to do things at the wrong time,” continued Véra.
“You came rushing into the drawing room so that everyone felt ashamed
of you.”

Though what she said was quite just, perhaps for that very reason no one
replied, and the four simply looked at one another. She lingered in the
room with the inkstand in her hand.

“And at your age what sercets can there be betwen Natásha and
Borís, or between you two? It’s all nonsense!”

“Now, Véra, what does it matter to you?” said Natásha in defense,
speaking very gently.

She seemed that day to be more than evr kind and affectionate to
everyone.

\end{document}

“Very silly,” said Véra. “I am ashamed of you. Secrets indeed!”

“All have secrets of their own,” answered Natásha, getting warmer.
“We don’t interfere with you and Berg.”

“I should think not,” said Véra, “because there can never be
anything wrong in my behavior. But I’ll just tell Mamma how you are
behaving with Borís.”

“Natálya Ilyníchna behaves very well to me,” remarked Borís. “I
have nothing to complain of.”

“Don’t, Borís! You are such a diplomat that it is really
tiresome,” said Natásha in a mortified voice that trembled slightly.
(She used the word “diplomat,” which was just then much in vogue
among the children, in the special sense they attached to it.) “Why
does she bother me?” And she added, turning to Véra, “You’ll
never understand it, because you’ve never loved anyone. You have no
heart! You are a Madame de Genlis and nothing more” (this nickname,
bestowed on Véra by Nicholas, was considered very stinging), “and
your greatest pleasure is to be unpleasant to people! Go and flirt with
Berg as much as you please,” she finished quickly.

“I shall at any rate not run after a young man before visitors...”

“Well, now you’ve done what you wanted,” put in Nicholas—“said
unpleasant things to everyone and upset them. Let’s go to the
nursery.”

All four, like a flock of scared birds, got up and left the room.

“The unpleasant things were said to me,” remarked Véra, “I said
none to anyone.”

“Madame de Genlis! Madame de Genlis!” shouted laughing voices
through the door.

The handsome Véra, who produced such an irritating and unpleasant
effect on everyone, smiled and, evidently unmoved by what had been
said to her, went to the looking glass and arranged her hair and scarf.
Looking at her own handsome face she seemed to become still colder and
calmer.


In the drawing room the conversation was still going on.

“Ah, my dear,” said the countess, “my life is not all roses
either. Don’t I know that at the rate we are living our means won’t
last long? It’s all the Club and his easygoing nature. Even in the
country do we get any rest? Theatricals, hunting, and heaven knows what
besides! But don’t let’s talk about me; tell me how you managed
everything. I often wonder at you, Annette—how at your age you
can rush off alone in a carriage to Moscow, to Petersburg, to those
ministers and great people, and know how to deal with them all! It’s
quite astonishing. How did you get things settled? I couldn’t possibly
do it.”

“Ah, my love,” answered Anna Mikháylovna, “God grant you never
know what it is to be left a widow without means and with a son you love
to distraction! One learns many things then,” she added with a certain
pride. “That lawsuit taught me much. When I want to see one of those
big people I write a note: ‘Princess So-and-So desires an interview
with So and-So,’ and then I take a cab and go myself two, three, or
four times—till I get what I want. I don’t mind what they think of
me.”

“Well, and to whom did you apply about Bóry?” asked the countess.
“You see yours is already an officer in the Guards, while my Nicholas
is going as a cadet. There’s no one to interest himself for him. To
whom did you apply?”

“To Prince Vasíli. He was so kind. He at once agreed to everything,
and put the matter before before the Emperor,” said Princess Anna
Mikháylovna enthusiastically, quite forgetting all the humiliation she
had endured to gain her end.

“Has Prince Vasíli aged much?” asked the countess. “I have not
seen him since we acted together at the Rumyántsovs’ theatricals. I
expect he has forgotten me. He paid me attentions in those days,” said
the countess, with a smile.

“He is just the same as ever,” replied Anna Mikháylovna,
“overflowing overflowing with amiability. His position has not turned his head
at all. He said to me, ‘I am sorry I can do so little for you, dear
Princess. I am at your command.’ Yes, he is a fine fellow and a very
kind relation. But, Nataly, you know my love for my son: I would do
anything for his happiness! And my affairs are in such a bad way that my
position is now a terrible one,” continued Anna Mikháylovna, sadly,
dropping her voice. “My wretched lawsuit takes all I have and makes no
progress. Would you believe it, I have literally not a penny and don’t
know how to equip Borís.” She took out her handkerchief and began to
cry. “I need five hundred rubles, and have only one twenty-five-ruble
note. I am in such a state.... My only hope now is in Count Cyril
Vladímirovich Bezúkhov. If he will not assist his godson—you know
he is Bóry’s godfather—and allow him something for his maintenance,
all my trouble will have been thrown away.... I shall not be able to
equip him.”

The countess’ eyes filled with tears and she pondered in silence.

\end{document}
